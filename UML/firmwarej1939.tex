\begin {sequencediagram}
\newthread {main}{Main}


\newinst [1]{serial}{PuertoSerial}
\newinst [0]{name}{Init}
\newinst [1]{can}{BufferCAN}

\newinst [0]{pgn}{PGN}
\newinst [0]{int}{interrupción}
 

%%%%%enlaces%%%%%

\begin{call}{main}{J1939init()}{name}{true}
	\begin{call} 
		{name}{address claim}{can}{true-false}
	\end{call}
\end{call}

\begin{call}{main}{timer2()}{name}{true}
\end{call}

\begin{call}{main}{serial-init()}{name}{true}
\end{call}

\begin{call}{main}{serial-isr()}{int}{return PID}
\end{call}


\begin{sdblock}[blue!10]{Loop}{}

	\begin{call}[2]{main}{j1939GetMessage()}{can}{return data}
	\end{call}
	
	\begin{call}[2]{main}{lecturaParametro}{pgn}{true}\end{call}
    
    \begin{call}[2]{main}{print(JSON)}{serial}{void}
	\end{call}
    
    
	\end{sdblock}
\end {sequencediagram}